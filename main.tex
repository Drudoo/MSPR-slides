\documentclass{beamer}

% There are many different themes available for Beamer. A comprehensive
% list with examples is given here:
% http://deic.uab.es/~iblanes/beamer_gallery/index_by_theme.html
% You can uncomment the themes below if you would like to use a different
% one:
%\usetheme{AnnArbor}
%\usetheme{Antibes}
%\usetheme{Bergen}
%\usetheme{Berkeley}
%\usetheme{Berlin}
%\usetheme{Boadilla}
%\usetheme{boxes}
%\usetheme{CambridgeUS}
\usetheme{Copenhagen}
%\usetheme{Darmstadt}
%\usetheme{default}
%\usetheme{Frankfurt}
%\usetheme{Goettingen}
%\usetheme{Hannover}
%\usetheme{Ilmenau}
%\usetheme{JuanLesPins}
%\usetheme{Luebeck}
%\usetheme{Madrid}
%\usetheme{Malmoe}
%\usetheme{Marburg}
%\usetheme{Montpellier}
%\usetheme{PaloAlto}
%\usetheme{Pittsburgh}
%\usetheme{Rochester}
%\usetheme{Singapore}
%\usetheme{Szeged}
%\usetheme{Warsaw}
\usepackage{todonotes}
\title{Multivariate Statistics and and Pattern Recognition}

% A subtitle is optional and this may be deleted
\subtitle{Exam Presentation - Skin Segmentation}

\author{Frederik~S.~Vanggaard~B\ae rentsen}
% - Give the names in the same order as the appear in the paper.
% - Use the \inst{?} command only if the authors have different
%   affiliation.

\institute[Aalborg University - Copenhagen] % (optional, but mostly needed)
{
  School of Information, Communication and Technology\\
  Aalborg University - Copenhagen}
% - Use the \inst command only if there are several affiliations.
% - Keep it simple, no one is interested in your street address.

\date{Fall 2014}
% - Either use conference name or its abbreviation.
% - Not really informative to the audience, more for people (including
%   yourself) who are reading the slides online

\subject{Multivariate Statistics and and Pattern Recognition}
% This is only inserted into the PDF information catalog. Can be left
% out. 

% If you have a file called "university-logo-filename.xxx", where xxx
% is a graphic format that can be processed by latex or pdflatex,
% resp., then you can add a logo as follows:

% \pgfdeclareimage[height=0.5cm]{university-logo}{university-logo-filename}
% \logo{\pgfuseimage{university-logo}}

% Delete this, if you do not want the table of contents to pop up at
% the beginning of each subsection:
\AtBeginSubsection[]
{
  \begin{frame}<beamer>{Outline}
    \tableofcontents[currentsection,currentsubsection]
  \end{frame}
}

% Let's get started
\begin{document}

\begin{frame}
  \titlepage
\end{frame}

\begin{frame}{Outline}
  \tableofcontents
  % You might wish to add the option [pausesections]
\end{frame}


%------------------NOTES---------------%
% - Different Classifiers (pdf)
%   - Parametric 
%   - Non-parametric
% - Cross-validation (180, 312, 332)
% - Feature reduction (216) / Selection (185, 195)
% - MATLAB code in appendix + print + usb stick
% - 
%--------------------------------------%


% Section and subsections will appear in the presentation overview
% and table of contents.
\section{Define the problem}

\subsection{What to classify?}

\begin{frame}{What to classify}{Skin Segmentation}
  \begin{itemize}
  \item {
    Data set provides 4 attributes
  }
  \item {
    Skin or Nonskin
    \pause
    \begin{itemize}
        \item Predict Skin or Nonskin based on \texttt{'R G B'}. 
        \pause
        \begin{itemize}
            \item Skin: \texttt{74 85 123 1}
            \item Nonskin: \texttt{61 62 18 2}
        \end{itemize}
    \end{itemize}
  }
  \end{itemize}
\end{frame}

\section{Classification}

\section{To Do}
\begin{frame}{To Do}
    \begin{itemize}
        \item Cumulative Eigenvalues(pcam)
    \end{itemize}
\end{frame}
\end{document}


